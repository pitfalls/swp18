
%%%%%%%%%%%%%%%%%%%%%%% file typeinst.tex %%%%%%%%%%%%%%%%%%%%%%%%%
%
% This is the  LaTeX source for the instructions to authors using
% the LaTeX document class 'llncs.cls' for contributions to
% the Lecture Notes in Computer Sciences series.
% http://www.springer.com/lncs       Springer Heidelberg 2006/05/04
%
% It may be used as a template for your own input - copy it
% to a new file with a new name and use it as the basis
% for your article.
%
% NB: the document class 'llncs' has its own and detailed documentation, see
% ftp://ftp.springer.de/data/pubftp/pub/tex/latex/llncs/latex2e/llncsdoc.pdf
%
%%%%%%%%%%%%%%%%%%%%%%%%%%%%%%%%%%%%%%%%%%%%%%%%%%%%%%%%%%%%%%%%%%%


\documentclass[12pt,runningheads,a4paper]{llncs}


\usepackage{amssymb}
\setcounter{tocdepth}{3}
\usepackage{graphicx}
\usepackage[mathcal]{eucal}

\usepackage[utf8]{inputenc}
\usepackage{hyperref}
\usepackage{mathtools}

\usepackage{array}
\usepackage{amsmath}
\usepackage{float}
\usepackage{soul}





\usepackage{listings}
\lstdefinelanguage{scala}{
  morekeywords={abstract,case,catch,class,def,%
    do,else,extends,false,final,finally,%
    for,if,implicit,import,match,mixin,%
    new,null,object,override,package,%
    private,protected,requires,return,sealed,%
    super,this,throw,trait,true,try,%
    type,val,var,while,with,yield},
  otherkeywords={=>,<-,<\%,<:,>:,\#,@},
  sensitive=true,
  morecomment=[l]{//},
  morecomment=[n]{/*}{*/},
  morestring=[b]",
  morestring=[b]',
  morestring=[b]"""
}

\lstset{
    literate={~} {$\sim$}{1}
}



\makeatletter
\renewcommand\chapter{\thispagestyle{plain}%
\global\@topnum\z@
\@afterindentfalse
\secdef\@chapter\@schapter}
\makeatother
\urldef{\mailsa}\path|alexander.frewein@student.tugraz.at|
\urldef{\mailsf}\path|fabian.fruehwirth@student.tugraz.at|
\urldef{\mailsp}\path|stephany.amizic@student.tugraz.at|
\topmargin-0.5cm


\begin{document}


% first the title is needed
\title{SWP Assignment 3}

% a short form should be given in case it is too long for the running head
\titlerunning{SWP Assignment 3}

% the name(s) of the author(s) follow(s) next
%
% NB: Chinese authors should write their first names(s) in front of
% their surnames. This ensures that the names appear correctly in
% the running heads and the author index.
%
\author{Alexander Frewein (01430019)\\
		Klaus Fabian Frühwirt (01131523)\\
		Stephany Amizic (01331786)}

%
\authorrunning{SWP Assigmnment 3}
% (feature abused for this document to repeat the title also on left hand pages)
% the affiliations are given next; don't give your e-mail address
% unless you accept that it will be published
\institute{Institute of Software Technology\\
\mailsa \\
\mailsf \\
\mailsp \\
}




\toctitle{SWP Assignment 3}
\tocauthor{Authors' Instructions}
\maketitle
\section*{Beispiel 1}
Codierung für Elemente aus A:\\
$\pi((n,d)) = build(n, build(d,[])) = [n,d]$ \\
\\
Codierung der Funktionen: \\
$\pi[mult](x, y) = build(first(rest(x)) \cdot first(rest(y)), build(f irst(x) \cdot first(y), []))$\\
$\pi[inverse](x) = build(first(rest(x)), build(first(x), [])). . . wenn first(x)\neq 0$\\
\\
Codierung des Prädikats:\\
$\pi[eq](x, y) = eq?(x, y)$\\
\\
\subsection*{Lösung}
Seien $n,d$ Elemente aus $Z$ und Nenner $\neq 0$, zB. $n=1, d=2$  \\
\\
Codierung für Elemente aus A:\\
$\pi((n,d)) = build(n, build(d,[])) = [n,d]$ \\
$\pi((n,d)) = [n,d]$\\
\\
Beispiel:\\
Linke Seite:\\
$\pi((1,2)) = build(1, build(2, [])) = build(1,[2]) = [1,2]$\\
\\
Rechte Seite:\\
$[n,d] = [1,2]$\\
\textbf{KORREKT}\\
\\
Codierung der Funktionen:\\
$\pi[mult](x, y) = build(first(rest(x)) \cdot first(rest(y)), build(first(x) \cdot first(y), []))$\\
$= build(first(rest(x))\cdot first(rest(y)), [first(x)\cdot first(y)]$\\
$= [first(rest(x))\cdot first(rest(y)), first(x)\cdot first(y)]$\\
$x = (a,b)$\\
$first (x) = a$\\
$first(rest(x)) = b$\\
\\
$y = (c,d)$\\
$first(y) = c$\\
$first(rest(y)) = d$\\
\\
$[a\cdot b, c\cdot d] = \pi(mult(x,y))= \pi (b\cdot d, a\cdot c)$\\
\\
Beispiel:\\
Seien $a, b, c, d$ Elemente aus $Z$ und Nenner $\neq 0$, zB. $a=1, b=2, c= 3, d=4$ und $x = (1,2), y = (3,4)$ \\
Linke Seite:\\
$\pi[mult]((1,2), (3,4)) = (2 \cdot 4, 1 \cdot 3) = (8, 3)$\\
Rechte Seite:\\
$build(2\cdot 4, build(1\cdot 3, []))= build(2\cdot 4, [1 \cdot 3])$\\
$= [2\cdot 4, 1 \cdot 3] = [8,3] $\\
\textbf{KORREKT}\\
\\
$\pi[inverse](x) = build(first(rest(x)), build(first(x), []))$\\
$= build(first(rest(x)),[first(x)])$\\
$= [first(rest(x)), first(x)]$\\
\\
$inverse(x) = (d,n)$\\
$x = (d,n)$\\
$first(rest(x)) = d$\\
$first(x) = n$\\
$\pi [inverse](x) = [d,n]$\\
$\pi [inverse](x)= \pi [(d,n)] = [d,n]$\\
\\
Beispiel:\\
Linke Seite:\\
$\pi [inverse](1,2) = (2,1)$\\
Rechte Seite: \\
$build(2, build(1, [])) = build(2, [1]) = [2,1]$\\
\textbf{KORREKT}\\
\\
Codierung des Prädikats:\\
$\pi[eq](x, y) = eq?(x, y)$\\
\\
Beispiel 1:\\
Linke Seite:\\
$\pi[eq]((1,2),(3,4)) \rightarrow \frac{1}{2} \neq \frac{3}{4} \rightarrow F$\\
Rechte Seite:\\
$eq?((1,2), (3,4)) = F$\\
Beispiel 2: \\
$a = 1, b = 1, c = 2, d = 2$\\
Linke Seite:\\
$\pi[eq]((1,1),(2,2)) \rightarrow \frac{1}{1} = \frac{2}{2} \rightarrow \frac{1}{1} = \frac{1}{1} \rightarrow F$\\
Rechte Seite:\\
$eq?((1,1), (2,2)) = T$\\
\textbf{KORREKT}\\
\section*{Beispiel 4}
\section*{Beispiel 2}
\subsection*{Codierung Prädikate:}
$\pi[isEmpty?] \ (x) = \\
if(emptyS(s)) \ then \ TRUE\\
else\\
FALSE\\
$
$
\\ \pi[isElement?] \ (x,y) = \\
if(eq?(first(x), y)) \ then \ TRUE\\
else\\
FALSE\\
$
\subsection*{Codierung Funktionen:}
$
\pi[insert] \ (x,y) = \\
if(isEmpty?(x)) \ then \\
build(y, []) \\
if(isElement?(x,y)) \ then \\
build(rest(x), []) \\
else \\
build(first(x), insert(rest(x), y)))\\
$
$
\\ \pi[remove] \ (x,y) = \\
if(isEmpty?(x)) \ then \\
build([],[]) \\
if(isElement?(x,y)) \ then \\
build(rest(x), []) \\
if(isEmpty?(rest(x))) \ then \\
build([], []) \\
else\\
build(first(x), remove(rest(x), y))
$
\subsection*{a)}
%&\rightarrow \forall \omega ', \omega' ~ x \ \omega ': 
\begin{flalign*}
& \exists x \ \forall y \ eq?(add(x,y),0) &\\
&Ip(\omega, \underline{(\exists x) \ (\forall y) \ eq?(add(x,y),0)})&\\
&\Leftrightarrow \exists \forall \omega' , \omega \ \sim \underline{x, y} \ \omega' : Ip(\omega', eq?(add(x,y),0)&\\
&\Leftrightarrow \exists \forall \omega' , \omega \ \sim \underline{x, y} \ \omega' : Ip(\omega', eq?(Ip(\omega', add(x, y), 0)) &\\
&\Leftrightarrow \exists \forall \omega' , \omega \ \sim \underline{x, y} \ \omega' : Ip(\omega', eq?(Ip(\omega', add(Ip(\omega', x), Ip(\omega', y), Ip(\omega', 0)) &\\
&\Leftrightarrow \exists \forall \omega' , \omega \ \sim \underline{x, y} \ \omega' : Ip(\omega', eq?(Ip(\omega', add(\omega(\underline{x}), \omega' (\underline{y})), \omega' (\underline{0}))) &\\
\end{flalign*}
Wir vermuten, dass der semantische Status gültig ist. Um dies zu beweisen müssen wir
eine Fallunterscheidung nach $\omega(y) \ und \ \omega(x)$ machen.\\
Fall 1:
\begin{flalign*}
&\omega(\underline{x}) = 0 \ \ \ \ \ \ \ \ \ \omega(\underline{y}) = 0&\\
&eq?(add(\omega(\underline{x}), \omega(\underline{y})), 0) &\\
&eq?(0,0) \rightarrow TRUE
\end{flalign*}
Fall 2:
\begin{flalign*}
&\omega(\underline{x}) = 1 \ \ \ \ \ \ \ \ \ \omega(\underline{y}) = 1&\\
&eq?(add(\omega(\underline{x}), \omega(\underline{y})), 0) &\\
&eq?(2,0) \rightarrow FALSE
\end{flalign*}
Der Ausdruck ist \textbf{erfüllbar}.
\newpage
\subsection*{b)}
\begin{flalign*}
& \exists a \ \lnot empty?(l) \rightarrow eq?(first(l), a) &\\
&Ip(\omega, (\exists a) \ \underline{\lnot empty?(l) \rightarrow eq?(first(l), a)})&\\
&\Leftrightarrow \exists \omega', \omega \ \sim \ \underline{a} \ \omega' : Ip(\omega', \lnot empty?(l) \rightarrow eq?(first(l), a))&\\
&\Leftrightarrow \exists \omega', \omega \ \sim \ \underline{a} \ \omega' : Ip(\omega', \lnot Ip(\omega', empty?(l)) \rightarrow eq?(Ip(\omega', first(l)), a))&\\
&\Leftrightarrow \exists \omega', \omega \ \sim \ \underline{a} \ \omega' : Ip(\omega', \lnot Ip(\omega', empty?(Ip(\omega', l)) \rightarrow eq?(Ip(\omega', first(Ip(\omega', l)), Ip(\omega', a)))&\\
&\Leftrightarrow \exists \omega', \omega \ \sim \ \underline{a} \ \omega' : Ip(\omega', \lnot Ip(\omega', empty?(\omega'(\underline{l})) \rightarrow eq?(Ip(\omega', first(\omega'(\underline{l})),\omega'(\underline{a})))&\\
\end{flalign*}
Wir vermuten, dass der semantische Status gültig ist. Um dies zu beweisen müssen wir
eine Fallunterscheidung nach $\omega(a) \ und \ \omega(l)$ machen.\\
Fall1:
\begin{flalign*}
&\omega(\underline{a}) == first(\omega(\underline{l}))&\\
&\lnot empty?(\omega(\underline{l})) = TRUE&\\
&eq?(first(\omega(\underline{l})), \omega(\underline{a})) = TRUE&\\
& TRUE \rightarrow TRUE = TRUE &\\
\end{flalign*}
Fall 2:
\begin{flalign*}
&\omega(\underline{a}) != first(\omega(\underline{l}))&\\
&\lnot empty?(\omega(\underline{l})) = TRUE&\\
&eq?(first(\omega(\underline{l})), \omega(\underline{a})) = FALSE&\\
& TRUE \rightarrow FALSE = FALSE &\\
\end{flalign*}
Der Ausdruck ist \textbf{erfüllbar}.
\subsection*{c)}
\begin{flalign*}
&\forall x \ eq?(mult(x, 2), y) \  \lor \ gt?(mult(x, x), y)&\\
&Ip(\omega, \underline{(\forall x) eq?(mult(x, 2), y) \  \lor \ gt?(mult(x, x), y)}) &\\
&\forall\omega', \omega \ \sim \ \underline{x} \ \omega' : Ip(\omega', eq?(mult(x, 2), y) \ \lor \ gt?(mult(x, x), y))&\\
&\forall\omega', \omega \ \sim \ \underline{x} \ \omega' : Ip(\omega', Ip(\omega' , eq?(mult(x, 2), y)) \ \lor \ Ip(\omega', gt?(mult(x, x), y)))&\\
&\forall\omega', \omega \ \sim \ \underline{x} \ \omega' : Ip(\omega', Ip(\omega' , eq?(Ip(\omega', mult(x, 2), y))) \ \lor \ Ip(\omega', gt?(Ip(\omega', mult(x, x), y))))&\\
&\forall\omega', \omega \ \sim \ \underline{x} \ \omega' : Ip(\omega', Ip(\omega' , eq?(Ip(\omega', mult(Ip(\omega' , x), Ip(\omega', 2)), Ip(\omega', y)))) \ \lor &\\
&\ \ \ \ \ \ \ \ \ \ \ \ \ \ \ \ \ \ \ \ \ \  Ip(\omega', gt?(Ip(\omega', mult(Ip(\omega', x), Ip(\omega', x)), Ip(\omega', y)))))&\\
&\forall\omega', \omega \ \sim \ \underline{x} \ \omega' : Ip(\omega', Ip(\omega' , eq?(Ip(\omega', mult(\omega'(\underline{x}), \omega'(\underline{2})), \omega'(\underline{y})))) \ \lor &\\
&\ \ \ \ \ \ \ \ \ \ \ \ \ \ \ \ \ \ \ \ \ \  Ip(\omega', gt?(Ip(\omega', mult(\omega'(\underline{x}), \omega'(\underline{x})), \omega'(\underline{y})))))&\\
\end{flalign*}
Wir vermuten, dass der semantische Status gültig ist. Um dies zu beweisen müssen wir
eine Fallunterscheidung nach $\omega(x)$ machen.\\
Fall 1:
\begin{flalign*}
&\omega(\underline{x}) = 0&\\
&eq?(mult(\omega(\underline{x}), 2), \omega(\underline{y}))&\\
&eq?(mult(0, 2), \omega(\underline{y}))&\\
&eq?(0, \omega(\underline{y}))&\\
&TRUE \ falls \ \omega(\underline{y}) = 0&\\
&FALSE \ falls \ \omega(\underline{y}) != 0&\\
&T \ \lor \ gt?(mult(\omega(\underline{x}), \omega(\underline{x})), \omega(\underline{y}))&\\
&T \ \lor \ gt?(mult(0, 0)), \omega(\underline{y}))&\\
&gt?(0, \omega(\underline{y}))  = FALSE&\\
&T \ \lor \ F \rightarrow TRUE&\\
&&\\
&F \ \lor \ gt?(mult(\omega(\underline{x}), \omega(\underline{x})), \omega(\underline{y}))&\\
&F \ \lor \ gt?(mult(0, 0)), \omega(\underline{y}))&\\
&gt?(0, \omega(\underline{y}))  = FALSE&\\
&F \ \lor \ F \rightarrow FALSE&\\
\end{flalign*}
Der Ausdruck ist \textbf{erfüllbar}.

\section*{Beispiel 3}
\begin{align}
&\indent\omega(a) = 42\\
&\indent I_\mathcal{A}(\omega, Z_{2-8}; a := a+a)\\
&\indent I_\mathcal{A}(I_\mathcal{A}(\omega, x := add(1,2); Z_{4-7})\ a := a + a)\\
&\indent I_\mathcal{A}( I_\mathcal{A}(I_\mathcal{A}(\omega, x := add(1,2))\ Z_{4-7})\ a := a + a)
\end{align}
\begin{gather*}
NE: \omega^1 \sim_x \omega,\: \omega^1(x) = I_\mathcal{T}(\omega, add(1,2)) = add(1,2) = 3
\end{gather*}
\begin{align}
&\indent I_\mathcal{A}(I_\mathcal{A}(\omega^1, Z_{4-7})\ a := a+a)\\
&\indent I_\mathcal{A}(I_\mathcal{A}(\omega^1, if\ eq?(x,3)\ then\ a := a + a\ else\ x := add(x,2))\ a := a+a)
\end{align}
\begin{gather*}
NR: I_\mathcal{P}(\omega^1, eq?(x,3)) = eq?(I_\mathcal{P}(\omega^1,x),3) = eq?(3,3) = \top
\end{gather*}
\begin{align}
&\indent I_\mathcal{A}(I_\mathcal{A}(\omega^1, a := a + a)\ a := a+a)
\end{align}
\begin{gather*}
NE: \omega^2 \sim_a \omega^1,\: \omega^2(a) = I_\mathcal{T}(\omega^2,a ) = (\omega^1,a )+(\omega^1,a )  = 6
\end{gather*}
\begin{align}
&\indent I_\mathcal{A}(\omega^2, a := a+a)
\end{align}
\begin{gather*}
NE: \omega^3 \sim_a \omega^2,\: \omega^3(a) = I_\mathcal{T}(\omega^3,a ) = (\omega^2,a )+(\omega^2,a )  = 12
\end{gather*}
\begin{align}
\underline{\omega^3}
\end{align}


\subsection*{b)}
Man vereinfache Programm 1, analog zu a.), weitestmöglich. Zu beachten ist außerdem, dass $\omega$ in diesem Fall 
über keine anfänglichen Belegungen verfügt.
\begin{align}
&\indent I_\mathcal{A}(I_\mathcal{A}(\omega, Z_{2-8});\ \alpha) \\
&\indent I_\mathcal{A}(I_\mathcal{A}(I_\mathcal{A}(\omega, x := add(1,2)); Z_{4-7});\ \alpha)
\end{align}
\begin{gather*}
NE: \omega^1 \sim_x \omega,\: \omega^1(x) = I_\mathcal{T}(\omega, add(1,2)) = add(1,2) = 3
\end{gather*}
\begin{align}
&\indent I_\mathcal{A}(I_\mathcal{A}(\omega^1, Z_{4-7});\ \alpha)\\
&\indent I_\mathcal{A}(I_\mathcal{A}(\omega^1, if\ eq?(x,3)\ then\ \alpha\ else\ x := add(x,2)) );\ \alpha)
\end{align}
\begin{gather*}
NR: I_\mathcal{P}(\omega^1, eq?(x,3)) = eq?(I_\mathcal{P}(\omega^1,x),3) = eq?(3,3) = \top
\end{gather*}
\begin{align}
&\indent \underline{I_\mathcal{A}(I_\mathcal{A}(\omega^1, \alpha) \alpha)}
\end{align}
Man vereinfache nun Programm 2:
\begin{align}
&\indent I_\mathcal{A}(\bar{\omega}, x := 3; Z_{3-6})\\
&\indent I_\mathcal{A}(I_\mathcal{A}(\bar{\omega}, x := 3) \alpha; \alpha)\\
\end{align}
\begin{gather*}
NE: \bar{\omega}^1 \sim_x \omega,\: \bar{\omega}^1(x) = I_\mathcal{T}(\omega, 3) = 3
\end{gather*}
\begin{align}
&\indent \underline{I_\mathcal{A}(I_\mathcal{A}(\bar{\omega}^1, \alpha) \alpha)}
\end{align}
Damit wurde gezeigt, dass sich die Interpretationsfunktionen beider Programme ohne Einschränkung der Allgemeinheit 
auf dieselbe Funktion vereinfachen lassen. Desweiteren ist trivial zu erkennen, dass $\omega^1$ und $\bar{\omega}^1$ ident sind, 
womit die Gleichheit bewiesen ist.

\end{document}


