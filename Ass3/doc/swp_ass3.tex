
%%%%%%%%%%%%%%%%%%%%%%% file typeinst.tex %%%%%%%%%%%%%%%%%%%%%%%%%
%
% This is the  LaTeX source for the instructions to authors using
% the LaTeX document class 'llncs.cls' for contributions to
% the Lecture Notes in Computer Sciences series.
% http://www.springer.com/lncs       Springer Heidelberg 2006/05/04
%
% It may be used as a template for your own input - copy it
% to a new file with a new name and use it as the basis
% for your article.
%
% NB: the document class 'llncs' has its own and detailed documentation, see
% ftp://ftp.springer.de/data/pubftp/pub/tex/latex/llncs/latex2e/llncsdoc.pdf
%
%%%%%%%%%%%%%%%%%%%%%%%%%%%%%%%%%%%%%%%%%%%%%%%%%%%%%%%%%%%%%%%%%%%


\documentclass[12pt,runningheads,a4paper]{llncs}


\usepackage{amssymb}
\setcounter{tocdepth}{3}
\usepackage{graphicx}
\usepackage[mathcal]{eucal}

\usepackage[utf8]{inputenc}
\usepackage{hyperref}
\usepackage{mathtools}

\usepackage{array}
\usepackage{amsmath}
\usepackage{float}
\usepackage{soul}





\usepackage{listings}
\lstdefinelanguage{scala}{
  morekeywords={abstract,case,catch,class,def,%
    do,else,extends,false,final,finally,%
    for,if,implicit,import,match,mixin,%
    new,null,object,override,package,%
    private,protected,requires,return,sealed,%
    super,this,throw,trait,true,try,%
    type,val,var,while,with,yield},
  otherkeywords={=>,<-,<\%,<:,>:,\#,@},
  sensitive=true,
  morecomment=[l]{//},
  morecomment=[n]{/*}{*/},
  morestring=[b]",
  morestring=[b]',
  morestring=[b]"""
}

\lstset{
    literate={~} {$\sim$}{1}
}



\makeatletter
\renewcommand\chapter{\thispagestyle{plain}%
\global\@topnum\z@
\@afterindentfalse
\secdef\@chapter\@schapter}
\makeatother
\urldef{\mailsa}\path|alexander.frewein@student.tugraz.at|
\urldef{\mailsf}\path|fabian.fruehwirth@student.tugraz.at|
\urldef{\mailsp}\path|stephany.amizic@student.tugraz.at|
\topmargin-0.5cm


\begin{document}


% first the title is needed
\title{SWP Assignment 3}

% a short form should be given in case it is too long for the running head
\titlerunning{SWP Assignment 3}

% the name(s) of the author(s) follow(s) next
%
% NB: Chinese authors should write their first names(s) in front of
% their surnames. This ensures that the names appear correctly in
% the running heads and the author index.
%
\author{Alexander Frewein (01430019)\\
		Klaus Fabian Frühwirt (01131523)\\
		Stephany Amizic (01331786)}

%
\authorrunning{SWP Assigmnment 3}
% (feature abused for this document to repeat the title also on left hand pages)
% the affiliations are given next; don't give your e-mail address
% unless you accept that it will be published
\institute{Institute of Software Technology\\
\mailsa \\
\mailsf \\
\mailsp \\
}




\toctitle{SWP Assignment 3}
\tocauthor{Authors' Instructions}
\maketitle
\section*{Beispiel 4}
\subsection*{a)}
%&\rightarrow \forall \omega ', \omega' ~ x \ \omega ': 
\begin{flalign*}
& \exists x \ \forall y \ eq?(add(x,y),0) &\\
&Ip(\omega, \underline{(\exists x) \ (\forall y) \ eq?(add(x,y),0)})&\\
&\Leftrightarrow \exists \forall \omega' , \omega \ \sim \underline{x, y} \ \omega' : Ip(\omega', eq?(add(x,y),0)&\\
&\Leftrightarrow \exists \forall \omega' , \omega \ \sim \underline{x, y} \ \omega' : Ip(\omega', eq?(Ip(\omega', add(x, y), 0)) &\\
&\Leftrightarrow \exists \forall \omega' , \omega \ \sim \underline{x, y} \ \omega' : Ip(\omega', eq?(Ip(\omega', add(Ip(\omega', x), Ip(\omega', y), Ip(\omega', 0)) &\\
&\Leftrightarrow \exists \forall \omega' , \omega \ \sim \underline{x, y} \ \omega' : Ip(\omega', eq?(Ip(\omega', add(\omega(\underline{x}), \omega' (\underline{y})), \omega' (\underline{0}))) &\\
\end{flalign*}
Wir vermuten, dass der semantische Status gültig ist. Um dies zu beweisen müssen wir
eine Fallunterscheidung nach $\omega(y) \ und \ \omega(x)$ machen.\\
Fall 1:
\begin{flalign*}
&\omega(\underline{x}) = 0 \ \ \ \ \ \ \ \ \ \omega(\underline{y}) = 0&\\
&eq?(add(\omega(\underline{x}), \omega(\underline{y})), 0) &\\
&eq?(0,0) \rightarrow TRUE
\end{flalign*}
Fall 2:
\begin{flalign*}
&\omega(\underline{x}) = 1 \ \ \ \ \ \ \ \ \ \omega(\underline{y}) = 1&\\
&eq?(add(\omega(\underline{x}), \omega(\underline{y})), 0) &\\
&eq?(2,0) \rightarrow FALSE
\end{flalign*}
Der Ausdruck ist \textbf{erfüllbar}.
\newpage
\subsection*{b)}
\begin{flalign*}
& \exists a \ \lnot empty?(l) \rightarrow eq?(first(l), a) &\\
&Ip(\omega, (\exists a) \ \underline{\lnot empty?(l) \rightarrow eq?(first(l), a)})&\\
&\Leftrightarrow \exists \omega', \omega \ \sim \ \underline{a} \ \omega' : Ip(\omega', \lnot empty?(l) \rightarrow eq?(first(l), a))&\\
&\Leftrightarrow \exists \omega', \omega \ \sim \ \underline{a} \ \omega' : Ip(\omega', \lnot Ip(\omega', empty?(l)) \rightarrow eq?(Ip(\omega', first(l)), a))&\\
&\Leftrightarrow \exists \omega', \omega \ \sim \ \underline{a} \ \omega' : Ip(\omega', \lnot Ip(\omega', empty?(Ip(\omega', l)) \rightarrow eq?(Ip(\omega', first(Ip(\omega', l)), Ip(\omega', a)))&\\
&\Leftrightarrow \exists \omega', \omega \ \sim \ \underline{a} \ \omega' : Ip(\omega', \lnot Ip(\omega', empty?(\omega'(\underline{l})) \rightarrow eq?(Ip(\omega', first(\omega'(\underline{l})),\omega'(\underline{a})))&\\
\end{flalign*}
Wir vermuten, dass der semantische Status gültig ist. Um dies zu beweisen müssen wir
eine Fallunterscheidung nach $\omega(a) \ und \ \omega(l)$ machen.\\
Fall1:
\begin{flalign*}
&\omega(\underline{a}) == first(\omega(\underline{l}))&\\
&\lnot empty?(\omega(\underline{l})) = TRUE&\\
&eq?(first(\omega(\underline{l})), \omega(\underline{a})) = TRUE&\\
& TRUE \rightarrow TRUE = TRUE &\\
\end{flalign*}
Fall 2:
\begin{flalign*}
&\omega(\underline{a}) != first(\omega(\underline{l}))&\\
&\lnot empty?(\omega(\underline{l})) = TRUE&\\
&eq?(first(\omega(\underline{l})), \omega(\underline{a})) = FALSE&\\
& TRUE \rightarrow FALSE = FALSE &\\
\end{flalign*}
Der Ausdruck ist \textbf{erfüllbar}.
\subsection*{c)}
\begin{flalign*}
&\forall x \ eq?(mult(x, 2), y) \  \lor \ gt?(mult(x, x), y)&\\
&Ip(\omega, \underline{(\forall x) eq?(mult(x, 2), y) \  \lor \ gt?(mult(x, x), y)}) &\\
&\forall\omega', \omega \ \sim \ \underline{x} \ \omega' : Ip(\omega', eq?(mult(x, 2), y) \ \lor \ gt?(mult(x, x), y))&\\
&\forall\omega', \omega \ \sim \ \underline{x} \ \omega' : Ip(\omega', Ip(\omega' , eq?(mult(x, 2), y)) \ \lor \ Ip(\omega', gt?(mult(x, x), y)))&\\
&\forall\omega', \omega \ \sim \ \underline{x} \ \omega' : Ip(\omega', Ip(\omega' , eq?(Ip(\omega', mult(x, 2), y))) \ \lor \ Ip(\omega', gt?(Ip(\omega', mult(x, x), y))))&\\
&\forall\omega', \omega \ \sim \ \underline{x} \ \omega' : Ip(\omega', Ip(\omega' , eq?(Ip(\omega', mult(Ip(\omega' , x), Ip(\omega', 2)), Ip(\omega', y)))) \ \lor &\\
&\ \ \ \ \ \ \ \ \ \ \ \ \ \ \ \ \ \ \ \ \ \  Ip(\omega', gt?(Ip(\omega', mult(Ip(\omega', x), Ip(\omega', x)), Ip(\omega', y)))))&\\
&\forall\omega', \omega \ \sim \ \underline{x} \ \omega' : Ip(\omega', Ip(\omega' , eq?(Ip(\omega', mult(\omega'(\underline{x}), \omega'(\underline{2})), \omega'(\underline{y})))) \ \lor &\\
&\ \ \ \ \ \ \ \ \ \ \ \ \ \ \ \ \ \ \ \ \ \  Ip(\omega', gt?(Ip(\omega', mult(\omega'(\underline{x}), \omega'(\underline{x})), \omega'(\underline{y})))))&\\
\end{flalign*}
Wir vermuten, dass der semantische Status gültig ist. Um dies zu beweisen müssen wir
eine Fallunterscheidung nach $\omega(x)$ machen.\\
Fall 1:
\begin{flalign*}
&\omega(\underline{x}) = 0&\\
&eq?(mult(\omega(\underline{x}), 2), \omega(\underline{y}))&\\
&eq?(mult(0, 2), \omega(\underline{y}))&\\
&eq?(0, \omega(\underline{y}))&\\
&TRUE \ falls \ \omega(\underline{y}) = 0&\\
&FALSE \ falls \ \omega(\underline{y}) != 0&\\
&T \ \lor \ gt?(mult(\omega(\underline{x}), \omega(\underline{x})), \omega(\underline{y}))&\\
&T \ \lor \ gt?(mult(0, 0)), \omega(\underline{y}))&\\
&gt?(0, \omega(\underline{y}))  = FALSE&\\
&T \ \lor \ F \rightarrow TRUE&\\
&&\\
&F \ \lor \ gt?(mult(\omega(\underline{x}), \omega(\underline{x})), \omega(\underline{y}))&\\
&F \ \lor \ gt?(mult(0, 0)), \omega(\underline{y}))&\\
&gt?(0, \omega(\underline{y}))  = FALSE&\\
&F \ \lor \ F \rightarrow FALSE&\\
\end{flalign*}
Der Ausdruck ist \textbf{erfüllbar}.
\end{document}

