
%%%%%%%%%%%%%%%%%%%%%%% file typeinst.tex %%%%%%%%%%%%%%%%%%%%%%%%%
%
% This is the  LaTeX source for the instructions to authors using
% the LaTeX document class 'llncs.cls' for contributions to
% the Lecture Notes in Computer Sciences series.
% http://www.springer.com/lncs       Springer Heidelberg 2006/05/04
%
% It may be used as a template for your own input - copy it
% to a new file with a new name and use it as the basis
% for your article.
%
% NB: the document class 'llncs' has its own and detailed documentation, see
% ftp://ftp.springer.de/data/pubftp/pub/tex/latex/llncs/latex2e/llncsdoc.pdf
%
%%%%%%%%%%%%%%%%%%%%%%%%%%%%%%%%%%%%%%%%%%%%%%%%%%%%%%%%%%%%%%%%%%%


\documentclass[12pt,runningheads,a4paper]{llncs}


\usepackage{amssymb}
\setcounter{tocdepth}{3}
\usepackage{graphicx}
\usepackage[mathcal]{eucal}

\usepackage[utf8]{inputenc}
\usepackage{hyperref}
\usepackage{mathtools}

\usepackage{array}
\usepackage{amsmath}
\usepackage{float}
\usepackage{soul}





\usepackage{listings}
\lstdefinelanguage{scala}{
  morekeywords={abstract,case,catch,class,def,%
    do,else,extends,false,final,finally,%
    for,if,implicit,import,match,mixin,%
    new,null,object,override,package,%
    private,protected,requires,return,sealed,%
    super,this,throw,trait,true,try,%
    type,val,var,while,with,yield},
  otherkeywords={=>,<-,<\%,<:,>:,\#,@},
  sensitive=true,
  morecomment=[l]{//},
  morecomment=[n]{/*}{*/},
  morestring=[b]",
  morestring=[b]',
  morestring=[b]"""
}

\lstset{
    literate={~} {$\sim$}{1}
}



\makeatletter
\renewcommand\chapter{\thispagestyle{plain}%
\global\@topnum\z@
\@afterindentfalse
\secdef\@chapter\@schapter}
\makeatother
\urldef{\mailsa}\path|alexander.frewein@student.tugraz.at|
\urldef{\mailsf}\path|fabian.fruehwirth@student.tugraz.at|
\urldef{\mailsp}\path|stephany.amizic@student.tugraz.at|
\topmargin-0.5cm


\begin{document}
\newcolumntype{L}[1]{>{\raggedright\arraybackslash}p{#1}}
\newcolumntype{C}[1]{>{\centering\arraybackslash}p{#1}}
\newcolumntype{R}[1]{>{\raggedleft\arraybackslash}p{#1}}

% first the title is needed
\title{SWP Assignment 4}

% a short form should be given in case it is too long for the running head
\titlerunning{SWP Assignment 4}

% the name(s) of the author(s) follow(s) next
%
% NB: Chinese authors should write their first names(s) in front of
% their surnames. This ensures that the names appear correctly in
% the running heads and the author index.
%
\author{Alexander Frewein (01430019)\\
		Klaus Fabian Frühwirt (01131523)\\
		Stephany Amizic (01331786)}

%
\authorrunning{SWP Assigmnment 4}
% (feature abused for this document to repeat the title also on left hand pages)
% the affiliations are given next; don't give your e-mail address
% unless you accept that it will be published
\institute{Institute of Software Technology\\
\mailsa \\
\mailsf \\
\mailsp \\
}




\toctitle{SWP Assignment 4}
\tocauthor{Authors' Instructions}
\maketitle

\section*{Beispiel 2}
nat(0)\\
nat(s(A)) : -nat(A)\\
leq(0, s(Y)) : - nat(Y)\\
leq(s(x), s(y)): - leq(X,Y)\\

\begin{tabular}{|C{4cm}|L{6cm}|L{6cm}|}\hline
   C1 & n(0) &\\ \hline
   C2 &  $n(s(A))  \lor   \sim  n(A)$ & \\ \hline
   C3 & $leq(0, s(Y))  \lor   \sim   nat(Y)$ & \\ \hline
   C4 & $leq(s(X), s(Y))  \lor  \sim leq(X,Y)$ & \\ \hline
   C5 & $\sim leq(s(s(0)), s(s(s(0))))$ & \\ \hline
    &  & \\ \hline
    & $\sim leq(s(s(0)), s(s(s(0))))$ & $leq(s(X), s(Y)) \lor \sim leq(X,Y)$\\
    									& & A = s(0)\\ 
    									& & B = s(s(0)) \\ \hline
    & $\sim leq(s(0), s(s(0))) $	  & $leq(s(X), s(Y)) \lor \sim leq(X,Y)$\\
    								    & & A = 0\\
    								    & & B = s(0) \\ \hline
    & $\sim leq(0, s(0))	$		  & $leq(0, s(Y)) \lor \sim nat(Y)$\\
    								   & & B = 0 \\
    								   & & n(0) \\ \hline
    & $\sim nat(Y)$  & \\ \hline
    & [ ] & \\ \hline								   									    
    								     		
   \end{tabular}
\end{document}




