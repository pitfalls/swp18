
%%%%%%%%%%%%%%%%%%%%%%% file typeinst.tex %%%%%%%%%%%%%%%%%%%%%%%%%
%
% This is the  LaTeX source for the instructions to authors using
% the LaTeX document class 'llncs.cls' for contributions to
% the Lecture Notes in Computer Sciences series.
% http://www.springer.com/lncs       Springer Heidelberg 2006/05/04
%
% It may be used as a template for your own input - copy it
% to a new file with a new name and use it as the basis
% for your article.
%
% NB: the document class 'llncs' has its own and detailed documentation, see
% ftp://ftp.springer.de/data/pubftp/pub/tex/latex/llncs/latex2e/llncsdoc.pdf
%
%%%%%%%%%%%%%%%%%%%%%%%%%%%%%%%%%%%%%%%%%%%%%%%%%%%%%%%%%%%%%%%%%%%


\documentclass[12pt,runningheads,a4paper]{llncs}


\usepackage{amssymb}
\setcounter{tocdepth}{3}
\usepackage{graphicx}
\usepackage[mathcal]{eucal}

\usepackage[utf8]{inputenc}
\usepackage{hyperref}
\usepackage{mathtools}

\usepackage{array}
\usepackage{amsmath}
\usepackage{float}





\usepackage{listings}
\lstdefinelanguage{scala}{
  morekeywords={abstract,case,catch,class,def,%
    do,else,extends,false,final,finally,%
    for,if,implicit,import,match,mixin,%
    new,null,object,override,package,%
    private,protected,requires,return,sealed,%
    super,this,throw,trait,true,try,%
    type,val,var,while,with,yield},
  otherkeywords={=>,<-,<\%,<:,>:,\#,@},
  sensitive=true,
  morecomment=[l]{//},
  morecomment=[n]{/*}{*/},
  morestring=[b]",
  morestring=[b]',
  morestring=[b]"""
}

\lstset{
    literate={~} {$\sim$}{1}
}



\makeatletter
\renewcommand\chapter{\thispagestyle{plain}%
\global\@topnum\z@
\@afterindentfalse
\secdef\@chapter\@schapter}
\makeatother
\urldef{\mailsa}\path|alexander.frewein@student.tugraz.at|
\urldef{\mailsf}\path|fabian.fruehwirth@student.tugraz.at|
\urldef{\mailsp}\path|stephany.amizic@student.tugraz.at|
\topmargin-0.5cm


\begin{document}


% first the title is needed
\title{SWP Assignment 1}

% a short form should be given in case it is too long for the running head
\titlerunning{SWP Assignment 1}

% the name(s) of the author(s) follow(s) next
%
% NB: Chinese authors should write their first names(s) in front of
% their surnames. This ensures that the names appear correctly in
% the running heads and the author index.
%
\author{Alexander Frewein (01430019)\\
		Klaus Fabian Frühwirt (01131523)\\
		Stephany Amizic (01331786)}

%
\authorrunning{SWP Assigmnment 1}
% (feature abused for this document to repeat the title also on left hand pages)
% the affiliations are given next; don't give your e-mail address
% unless you accept that it will be published
\institute{Institute of Software Technology\\
\mailsa \\
\mailsf \\
\mailsp \\
}




\toctitle{SWP Assignment 1}
\tocauthor{Authors' Instructions}
\maketitle


\section*{Beispiel 1}
\subsection*{a.)}

\begin{align*}
L &= \{\underline{a}(\underline{a}\underline{a}|\underline{b})^* \underline{c}\}\\
S &\rightarrow \underline{a}A  \\
S &\rightarrow \underline{a}B  \\
S &\rightarrow \underline{a}C  \\
A &\rightarrow \underline{a}E  \\
A &\rightarrow \underline{c}D  \\
B &\rightarrow \underline{b}B  \\
B &\rightarrow \underline{c}D  \\
C &\rightarrow \underline{c}D  \\
D &\rightarrow \epsilon  \\
E &\rightarrow \underline{a}A
\end{align*}


\subsection*{b.)}
\begin{align*}
L &= \{\underline{a}^{(2n)} \underline{b} \; \underline{c}^* \; (\underline{bb} | \underline{d}) \; | n>0 \}\\
S &\rightarrow \underline{a}A  \\
A &\rightarrow \underline{a}A  \\
A &\rightarrow \underline{b}C  \\
A &\rightarrow \underline{b}B  \\
A &\rightarrow \underline{b}D  \\
B &\rightarrow \underline{b}E  \\
B &\rightarrow \epsilon  \\
C &\rightarrow \underline{c}C  \\
C &\rightarrow \underline{c}B  \\
C &\rightarrow \underline{c}D  \\
D &\rightarrow \underline{d}D  \\
D &\rightarrow \epsilon  \\
E &\rightarrow \underline{b}B
\end{align*}

\section*{Beispiel 2}
\subsection*{a.)}
Dies ist eine \textbf{allgemeine} Grammatik da $ \mid\alpha\mid \leq \mid\beta\mid$ \textbf{nicht} gilt und somit keine Restriktion $\alpha \rightarrow \beta$  gilt 
\subsection*{b.)}
Dies ist eine \textbf{reguläre} Grammatik da $ \mid\alpha\mid \leq \mid\beta\mid , \alpha \in$ V\textsubscript{N} 
$\beta$ hat form aA oder a, mit \\a $\in$ V\textsubscript{T} $\bigcup$ \{$\epsilon$\}, A $\in$ V\textsubscript{N}
\subsection*{c.)}
Dies ist eine \textbf{kontextfreie} Grammatik da  $\mid\alpha\mid \leq \mid\beta\mid\, , \alpha \in$ V\textsubscript{N}
\subsection*{d.)}
Dies ist keine \textbf{gültige} Grammatik da $R \rightarrow Q\underline{y} $ nicht laut Definition $\alpha \, \beta \in (V\textsubscript{N} \cup V\textsubscript{T})$ diese Form nicht in der Grammatik definiert ist.
\subsection*{e.)}
Dies ist eine \textbf{allgemeine} Grammatik da $ \mid\alpha\mid \leq \mid\beta\mid$ \textbf{nicht} gilt und somit keine Restriktion $\alpha \rightarrow \beta$  gilt 
\subsection*{f.)}
Dies ist keine \textbf{gültige} Grammatik da $\underline{num} \rightarrow      \underline{var}$ nicht in den Grammatik definiert ist



\section*{Beispiel 3}
\subsection*{First und Follow Mengen:}

\begin{table}[htbp]
    \centering
      \begin{tabular}{|m{1cm}|m{2cm}|m{2cm}|}
\hline
 &FIRST & FOLLOW \\ \hline
S & \underline{a} \underline{b} \underline{c} \underline{d} \underline{e} & \$ \\ \hline
A & \underline{a}  \underline{b} \underline{c} &  \underline{d} \underline{e} \\ \hline
B & \underline{d} \underline{e} & \underline{c} \underline{d} \underline{e}\\ \hline
C & \underline{b}  & \underline{c} \underline{d} \underline{e} \\ \hline
D & \underline{c} & \underline{d} \underline{e} \\ \hline
E & \underline{d} \underline{e} & \underline{c} \underline{d} \underline{e}\\ 
\hline
        \end{tabular}
\end{table}

\newpage
\subsection*{LL(1) Tabelle}

\begin{table}[htbp]
    \centering
      \begin{tabular}{|c|c|c|c|c|c|c|}
\hline
   & \underline{a} & \underline{b} & \underline{c} & \underline{d} & \underline{e} & \$ \\ \hline
S & S $\rightarrow$ AB & S $\rightarrow$ AB & S $\rightarrow$ AB&  S $\rightarrow$ AB  &  S $\rightarrow$ AB &  S $\rightarrow$ AB \\ \hline
A & A $\rightarrow$ \underline{a}A & A $\rightarrow$ CD & A $\rightarrow$ CD &   &  &  \\ \hline
B &  &   &   & B $\rightarrow$ E &B $\rightarrow$ E  &   \\ \hline
C  &  & C $\rightarrow$ \underline{b} B & C $\rightarrow \epsilon$ &C $\rightarrow \epsilon$   & C $\rightarrow \epsilon$ & \\ \hline
D &  &   &D $\rightarrow$ \underline{c}C   &D $\rightarrow \epsilon$   &D $\rightarrow \epsilon$  &   \\ \hline
E &  &   & &E $\rightarrow$ \underline{d}   &E $\rightarrow$ \underline{e}  &   \\
\hline
        \end{tabular}
\end{table}






\end{document}

