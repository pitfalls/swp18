
%%%%%%%%%%%%%%%%%%%%%%% file typeinst.tex %%%%%%%%%%%%%%%%%%%%%%%%%
%
% This is the  LaTeX source for the instructions to authors using
% the LaTeX document class 'llncs.cls' for contributions to
% the Lecture Notes in Computer Sciences series.
% http://www.springer.com/lncs       Springer Heidelberg 2006/05/04
%
% It may be used as a template for your own input - copy it
% to a new file with a new name and use it as the basis
% for your article.
%
% NB: the document class 'llncs' has its own and detailed documentation, see
% ftp://ftp.springer.de/data/pubftp/pub/tex/latex/llncs/latex2e/llncsdoc.pdf
%
%%%%%%%%%%%%%%%%%%%%%%%%%%%%%%%%%%%%%%%%%%%%%%%%%%%%%%%%%%%%%%%%%%%


\documentclass[12pt,runningheads,a4paper]{llncs}


\usepackage{amssymb}
\setcounter{tocdepth}{3}
\usepackage{graphicx}
\usepackage[mathcal]{eucal}

\usepackage[utf8]{inputenc}
\usepackage{hyperref}
\usepackage{mathtools}

\usepackage{array}
\usepackage{amsmath}
\usepackage{float}





\usepackage{listings}
\lstdefinelanguage{scala}{
  morekeywords={abstract,case,catch,class,def,%
    do,else,extends,false,final,finally,%
    for,if,implicit,import,match,mixin,%
    new,null,object,override,package,%
    private,protected,requires,return,sealed,%
    super,this,throw,trait,true,try,%
    type,val,var,while,with,yield},
  otherkeywords={=>,<-,<\%,<:,>:,\#,@},
  sensitive=true,
  morecomment=[l]{//},
  morecomment=[n]{/*}{*/},
  morestring=[b]",
  morestring=[b]',
  morestring=[b]"""
}

\lstset{
    literate={~} {$\sim$}{1}
}



\makeatletter
\renewcommand\chapter{\thispagestyle{plain}%
\global\@topnum\z@
\@afterindentfalse
\secdef\@chapter\@schapter}
\makeatother
\urldef{\mailsa}\path|alexander.frewein@student.tugraz.at|
\urldef{\mailsf}\path|fabian.fruehwirth@student.tugraz.at|
\topmargin-0.5cm


\begin{document}


% first the title is needed
\title{SWP Assigmnment 1}

% a short form should be given in case it is too long for the running head
\titlerunning{SWP Assigmnment 1}

% the name(s) of the author(s) follow(s) next
%
% NB: Chinese authors should write their first names(s) in front of
% their surnames. This ensures that the names appear correctly in
% the running heads and the author index.
%
\author{Alexander Frewein (1430019)\\
		Klaus Fabian Frühwirt (1131522)}
%
\authorrunning{SWP Assigmnment 1}
% (feature abused for this document to repeat the title also on left hand pages)
% the affiliations are given next; don't give your e-mail address
% unless you accept that it will be published
\institute{Institute of Software Technology\\
\mailsa \\
\mailsf \\
}

%
% NB: a more complex sample for affiliations and the mapping to the
% corresponding authors can be found in the file "llncs.dem"
% (search for the string "\mainmatter" where a contribution starts).
% "llncs.dem" accompanies the document class "llncs.cls".
%


%\begin{table}[htbp]
 %   \centering
   
    
  %    \begin{tabular}{|c|c|c|}
%\hline
 %&FIRST & FOLLOW \\ \hline
%S & Test & Test \\ \hline

 %       \end{tabular}
%\end{table}



\toctitle{SWP Assigmnment 1}
\tocauthor{Authors' Instructions}
\maketitle

\section*{Beispiel 1}
\subsection*{a.)}

\begin{align*}
L &= \{\underline{a}^* (\underline{c} |\underline{bb})^n \underline{a} | n>0	\}	\\
S &\rightarrow \underline{a}E  \\
S &\rightarrow \underline{c}C  \\
S &\rightarrow \underline{b}A  \\
A &\rightarrow \underline{b}B  \\
B &\rightarrow \underline{b}A  \\
B &\rightarrow \underline{a}D  \\
C &\rightarrow \underline{c}C  \\
C &\rightarrow \underline{a}D  \\
E &\rightarrow \underline{a}E  \\
E &\rightarrow \underline{c}C  \\
E &\rightarrow \underline{b}A  \\
D &\rightarrow \epsilon  \\
\end{align*}

\subsection*{b.)}

\begin{align*}
L &= \{(\underline{ad})^n \underline{b}^* \underline{c}^{(2n)} | n \geq 0    \}	\\
S &\rightarrow \underline{a}A \\
S &\rightarrow \underline{b}B \\
S &\rightarrow \underline{c}E \\
S &\rightarrow \epsilon \\
A &\rightarrow \underline{d}D \\
D &\rightarrow \underline{a}A \\
D &\rightarrow \underline{b}B \\
D &\rightarrow \underline{c}E \\
B &\rightarrow \underline{b}B \\
B &\rightarrow \underline{c}E \\
B &\rightarrow \epsilon \\
C &\rightarrow \underline{c}E \\
C &\rightarrow \epsilon \\
E &\rightarrow \underline{c}C \\
\end{align*}

\section*{Beispiel 2}

\subsection*{a.)}

\begin{align*}
S &\rightarrow AB \\
A &\rightarrow \underline{-} \\
A &\rightarrow \underline{+} \\
B &\rightarrow A \, \underline{var}\, B  \, \underline{var} \\
A &\rightarrow \epsilon \\
\end{align*}
Dies ist eine \textbf{kontextfreie} Grammatik da  $\mid\alpha\mid \leq \mid\beta\mid\, , \alpha \in$ V\textsubscript{N}
 


\subsection*{b.)}
\begin{align*}
S &\rightarrow \underline{private}\, P \\
\underline{private}\, P &\rightarrow P \\
\underline{private}\, P &\rightarrow \underline{public}\, P \\
P &\rightarrow \underline{int}\, Q \\
P &\rightarrow \underline{long}\, Q \\
Q &\rightarrow \underline{foo} \\
Q &\rightarrow \underline{bar}\\
\end{align*}
Dies ist eine \textbf{allgemeine} Grammatik da $ \mid\alpha\mid \leq \mid\beta\mid$ \textbf{nicht} gilt und somit keine Restriktion $\alpha \rightarrow \beta$  gilt 

\subsection*{c.)}
\begin{align*}
S &\rightarrow \underline{b}\, A \\
A &\rightarrow \underline{a}\, A \\
A &\rightarrow \underline{b}\, A \\
A &\rightarrow \underline{a}\\
A &\rightarrow \underline{b}\\
\end{align*}
Dies ist eine \textbf{reguläre} Grammatik da $ \mid\alpha\mid \leq \mid\beta\mid , \alpha \in$ V\textsubscript{N} 
$\beta$ hat form aA oder a, mit \\a $\in$ V\textsubscript{T} $\bigcup$ \{$\epsilon$\}, A $\in$ V\textsubscript{N}



\subsection*{d.)}
\begin{align*}
S &\rightarrow \underline{x}\, X \\
X &\rightarrow \underline{a}\, \underline{x} \\
\underline{x} &\rightarrow \underline{y} \underline{z} \\
Y &\rightarrow X\, Y \\
Y &\rightarrow \underline{z}\\
\end{align*}
Dies ist keine \textbf{gültige} Grammatik da $\underline{x} \rightarrow      \underline{y} \underline{z}$ nicht in den Grammatiken definiert ist


\subsection*{e.)}
\begin{align*}
S &\rightarrow Y\, S \, L \\
S &\rightarrow \underline{x}\\
Y &\rightarrow \underline{y}\, Y \\
Y &\rightarrow \underline{y}\\
L &\rightarrow \underline{l}\, L\\
L &\rightarrow \underline{l}\\
\end{align*}
Dies ist eine \textbf{kontextfreie} Grammatik da  $\mid\alpha\mid \leq \mid\beta\mid\, , \alpha \in$ V\textsubscript{N}

\subsection*{f.)}
\begin{align*}
S &\rightarrow \underline{num}\, X \\
S &\rightarrow \underline{var}\, X \\
\underline{var}\, X &\rightarrow \underline{var}\,  \underline{num}\, X \\
\underline{num}\, X &\rightarrow \underline{num}\, \underline{var}\, X \\
X &\rightarrow \underline{y}\\
\end{align*}
Dies ist eine \textbf{kontextsensitive} Grammatik da $\mid\alpha\mid \leq \mid\beta\mid$ gilt.

\section*{Beispiel 3}
\subsection*{a.)}
\subsection*{First und Follow Mengen:}

\begin{table}[htbp]
    \centering
      \begin{tabular}{|m{1cm}|m{2cm}|m{2cm}|}
\hline
 &FIRST & FOLLOW \\ \hline
S & \underline{x} \underline{var} \underline{val} \underline{y} & \$ \\ \hline
W & \underline{x} & \underline{var} \underline{val} \underline{y}\\ \hline
X & \underline{y} \underline{z} & \$ \\ \hline
Y & \underline{var} \underline{val} \underline{y} & \$ \\ \hline
Z & \underline{y} & \underline{y} \underline{z}\\ 
\hline
        \end{tabular}
\end{table}


\subsection*{LL(1) Tabelle}

\begin{table}[htbp]
    \centering
      \begin{tabular}{|c|c|c|c|c|c|c|}
\hline
   & \underline{y} & \underline{val} & \underline{var} & \underline{z} & \underline{x} & \$ \\ \hline
S & S $\rightarrow$ W Y & S $\rightarrow$ W Y & S $\rightarrow$ W Y &   & S $\rightarrow$ W Y & S $\rightarrow$ W Y \\ \hline
W & W $\rightarrow$ $\epsilon$ & W $\rightarrow$ $\epsilon$ & W $\rightarrow$ $\epsilon$ &   & W $\rightarrow$ \underline{x} W & W $\rightarrow$ $\epsilon$ \\ \hline
X & X $\rightarrow$ \underline{y} X &   &   & X $\rightarrow$ \underline{z} Z &  &   \\ \hline
Y  & Y $\rightarrow$ Z X & Y $\rightarrow$ val X & Y $\rightarrow$ \underline{var} X &   &  & Y $\rightarrow$ $\epsilon$\\ \hline
Z & Z $\rightarrow$ \underline{y} &   &   &   &  &   \\
\hline
        \end{tabular}
\end{table}

\section*{Beispiel4}
\subsection*{a.)}
\begin{table}[H]
    \centering
      \begin{tabular}{|m{2.5cm}|m{3.5cm}|m{3.5cm}|}
\hline
Stack & Input & Produktion \\ \hline
S \$& \underline{val} \underline{one} \underline{:}  \underline{String} \underline{=} \underline{"} \underline{0} \$& S $\rightarrow$ AB  \\ \hline
AB \$& \underline{val} \underline{one} \underline{:}  \underline{String} \underline{=} \underline{"} \underline{0} \$& A $\rightarrow$ CN \underline{:}\\ \hline
CN \underline{:} B \$& \underline{val} \underline{one} \underline{:}  \underline{String} \underline{=} \underline{"} \underline{0} \$& C $\rightarrow$ \underline{val} \\ \hline
\underline{val} N \underline{:} B \$& \underline{val} \underline{one} \underline{:}  \underline{String} \underline{=} \underline{"} \underline{0} \$& \underline{val} matched \\ \hline
N \underline{:} B \$& \underline{one} \underline{:}  \underline{String} \underline{=} \underline{"} \underline{0} \$& N $\rightarrow$ \underline{one}\\ \hline
\underline{one} \underline{:} B \$& \underline{one} \underline{:}  \underline{String} \underline{=} \underline{"} \underline{0} \$& \underline{one} matched\\ \hline
\underline{:} B \$& \underline{:}  \underline{String} \underline{=} \underline{"} \underline{0} \$& \underline{:} matched\\ \hline
 B \$& \underline{String} \underline{=} \underline{"} \underline{0} \$& B $\rightarrow$ \underline{String} \underline{=} \underline{"} V  \underline{"} \\ \hline
\underline{String} \underline{=} \underline{"} V  \underline{"} \$& \underline{String} \underline{=} \underline{"} \underline{0} \$& \underline{String} matched\\ \hline
\underline{=} \underline{"} V  \underline{"} \$&\underline{=} \underline{"} \underline{0} \$& \underline{=} matched\\ \hline
\underline{"} V  \underline{"} \$& \underline{"} \underline{0} \$& \underline{"} matched\\ \hline
V  \underline{"} \$& \underline{0} \$& V $\rightarrow$ \underline{0}\\ \hline
\underline{0}  \underline{"} \$& \underline{0} \$& \underline{0} matched\\ \hline
\underline{"} \$&\$ & NO MATCH\\
\hline
        \end{tabular}
\end{table}

\subsection*{b.)}

\begin{table}[H]
    \centering
      \begin{tabular}{|m{2cm}|m{3cm}|m{3cm}|}
\hline
Stack & Input & Produktion \\ \hline
S \$& \underline{var} \underline{two} \underline{:}  \underline{Int} \underline{=} \underline{1}\$& S $\rightarrow$ AB  \\ \hline
AB \$& \underline{var} \underline{two} \underline{:}  \underline{Int} \underline{=} \underline{1}\$& A $\rightarrow$ CN \underline{:} B  \\ \hline
CN \underline{:} B \$& \underline{var} \underline{two} \underline{:}  \underline{Int} \underline{=} \underline{1}\$&C $\rightarrow$ \underline{var} N \underline{:} \\ \hline
\underline{var} N \underline{:} B \$& \underline{var} \underline{two} \underline{:}  \underline{Int} \underline{=} \underline{1}\$& \underline{var} matched\\ \hline
\underline{two} \underline{:} B \$& \underline{two} \underline{:}  \underline{Int} \underline{=} \underline{1}\$&\underline{two} matched\\ \hline
\underline{:} B \$& \underline{:}  \underline{Int} \underline{=} \underline{1}\$&\underline{:} matched\\ \hline
B \$& \underline{Int} \underline{=} \underline{1}\$& B $\rightarrow$ \underline{Int} \underline{=} U\\ \hline
\underline{Int} \underline{=} U \$& \underline{Int} \underline{=} \underline{1}\$&\underline{Int}  matched\\ \hline
\underline{=} U \$& \underline{=} \underline{1}\$&\underline{=} matched\\ \hline
U \$&\underline{1}\$& U $\rightarrow$ \underline{1} V\\ \hline
\underline{1} V \$&\underline{1}\$& \underline{1} matched\\ \hline
V \$&\$&V $\rightarrow$ $\epsilon$ \\ \hline
\$ & \$ & MATCHED\\

\hline
        \end{tabular}
\end{table}


\section*{Beispiel 5}
\subsection*{\"Aquivalente LL(1) Grammatik:}
\begin{align*}
S &\rightarrow MN \\
T &\rightarrow  Q\underline{m}T \\
T &\rightarrow \epsilon \\
M &\rightarrow \underline{n} T \\
N &\rightarrow \underline{+} N \\
N &\rightarrow \underline{\#} \\
P &\rightarrow \underline{p} P \underline{p} \\
P &\rightarrow \underline{-} \\
Q &\rightarrow \underline{q} R \\
R &\rightarrow N \\
R &\rightarrow P \\
\end{align*}


\subsection*{First und Follow Mengen:}

\begin{table}[htbp]
    \centering
      \begin{tabular}{|m{1cm}|m{2cm}|m{2cm}|}
\hline
 &FIRST & FOLLOW \\ \hline
S&\underline{n}&\underline{\$}\\ \hline
M&\underline{n}&\underline{+} \underline{\#}\\ \hline
T&\underline{q}&\underline{+} \underline{\#}\\ \hline
Q&\underline{q}&\underline{m}\\ \hline
R&\underline{+} \underline{\#} \underline{p} \underline{-} & \underline{m}\\ \hline
N&\underline{+} \underline{\#}&\underline{m}\\ \hline
P&\underline{p} \underline{-}&\underline{p} \underline{m}\\ 
\hline
        \end{tabular}
\end{table}

\subsection*{LL(1) Tabelle}


\begin{table}[htbp]
    \centering
      \begin{tabular}{|c|c|c|c|c|c|c|m{1cm}|m{1cm}|}
\hline
    & \underline{q} & \underline{-} & \underline{p} & \underline{\#} & \underline{+} & \underline{n} & \underline{m}& \underline{\$}\\ \hline
S &   &   &   &   &   & S $\rightarrow$ M N & & \\ \hline
M &   &   &   &   &   & M $\rightarrow$ \underline{n} T &  &\\ \hline
N &   &   &   & N $\rightarrow$ \underline{\#} & N $\rightarrow$ + N &   & & \\ \hline
P &   & P $\rightarrow$ \underline{-} & P $\rightarrow$ \underline{p} P \underline{p} &   &   &  & &\\ \hline
Q & Q $\rightarrow$ \underline{q} R &   &   &   &   &   &  &\\ \hline
R &   & R $\rightarrow$ P & R $\rightarrow$ P & R $\rightarrow$ N & R $\rightarrow$ N &   & &\\ \hline
T & T $\rightarrow$ Q \underline{m} T &   &   & T $\rightarrow$ $\epsilon$ & T $\rightarrow$ $\epsilon$ & & &\\ 
\hline
        \end{tabular}
\end{table}

\newpage

\section*{Beispiel 6}
\subsection*{a.)}
\begin{lstlisting}[language=scala]
object SimpleParserA extends JavaTokenParsers{
  val A : Parser[Any] = "a"
  val B : Parser[Any] = "bb"
  val C : Parser[Any] =  "c"
  val BC: Parser[Any] = C | B

  val expression : Parser[Any] = opt(A) ~ rep(BC) ~ A

  def parse(s: String) = parseAll(expression, s)
}
\end{lstlisting}

\subsection*{b.)}

\begin{lstlisting}[language=scala]
object SimpleParserB extends JavaTokenParsers{
  val digit: Parser[String] = "0" | "1" | "2" | "3" |
                                    "4" | "5" | "6" |
                                    "7" | "8" | "9"
  val number : Parser[Any] = rep(digit)
  val operator : Parser[String] = "+" | "-" | "*" | "/"
  val equal : Parser[String] = "="
  
  val equation : Parser[Any] = number ~ operator ~ number ~ equal ~number
  
  def parse(s: String) = parseAll(equation, s)
}
\end{lstlisting}
\end{document}








