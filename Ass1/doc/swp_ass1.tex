
%%%%%%%%%%%%%%%%%%%%%%% file typeinst.tex %%%%%%%%%%%%%%%%%%%%%%%%%
%
% This is the  LaTeX source for the instructions to authors using
% the LaTeX document class 'llncs.cls' for contributions to
% the Lecture Notes in Computer Sciences series.
% http://www.springer.com/lncs       Springer Heidelberg 2006/05/04
%
% It may be used as a template for your own input - copy it
% to a new file with a new name and use it as the basis
% for your article.
%
% NB: the document class 'llncs' has its own and detailed documentation, see
% ftp://ftp.springer.de/data/pubftp/pub/tex/latex/llncs/latex2e/llncsdoc.pdf
%
%%%%%%%%%%%%%%%%%%%%%%%%%%%%%%%%%%%%%%%%%%%%%%%%%%%%%%%%%%%%%%%%%%%


\documentclass[12pt,runningheads,a4paper]{llncs}


\usepackage{amssymb}
\setcounter{tocdepth}{3}
\usepackage{graphicx}
\usepackage[mathcal]{eucal}

\usepackage[utf8]{inputenc}
\usepackage{hyperref}
\usepackage{mathtools}

\usepackage{array}
\usepackage{amsmath}
\usepackage{float}
\usepackage{soul}




\usepackage{listings}
\lstdefinelanguage{scala}{
  morekeywords={abstract,case,catch,class,def,%
    do,else,extends,false,final,finally,%
    for,if,implicit,import,match,mixin,%
    new,null,object,override,package,%
    private,protected,requires,return,sealed,%
    super,this,throw,trait,true,try,%
    type,val,var,while,with,yield},
  otherkeywords={=>,<-,<\%,<:,>:,\#,@},
  sensitive=true,
  morecomment=[l]{//},
  morecomment=[n]{/*}{*/},
  morestring=[b]",
  morestring=[b]',
  morestring=[b]"""
}

\lstset{
    literate={~} {$\sim$}{1}
}



\makeatletter
\renewcommand\chapter{\thispagestyle{plain}%
\global\@topnum\z@
\@afterindentfalse
\secdef\@chapter\@schapter}
\makeatother
\urldef{\mailsa}\path|alexander.frewein@student.tugraz.at|
\urldef{\mailsf}\path|fabian.fruehwirth@student.tugraz.at|
\urldef{\mailsp}\path|stephany.amizic@student.tugraz.at|
\topmargin-0.5cm


\begin{document}


% first the title is needed
\title{SWP Assignment 1}

% a short form should be given in case it is too long for the running head
\titlerunning{SWP Assignment 1}

% the name(s) of the author(s) follow(s) next
%
% NB: Chinese authors should write their first names(s) in front of
% their surnames. This ensures that the names appear correctly in
% the running heads and the author index.
%
\author{Alexander Frewein (01430019)\\
		Klaus Fabian Frühwirt (01131523)\\
		Stephany Amizic (01331786)}

%
\authorrunning{SWP Assigmnment 1}
% (feature abused for this document to repeat the title also on left hand pages)
% the affiliations are given next; don't give your e-mail address
% unless you accept that it will be published
\institute{Institute of Software Technology\\
\mailsa \\
\mailsf \\
\mailsp \\
}




\toctitle{SWP Assignment 1}
\tocauthor{Authors' Instructions}
\maketitle


\section*{Beispiel 1}
\subsection*{a.)}

\begin{align*}
L &= \{\underline{a}(\underline{a}\underline{a}|\underline{b})^* \underline{c}\}\\
S &\rightarrow \underline{a}A  \\
S &\rightarrow \underline{a}B  \\
S &\rightarrow \underline{a}C  \\
A &\rightarrow \underline{a}E  \\
A &\rightarrow \underline{c}D  \\
B &\rightarrow \underline{b}B  \\
B &\rightarrow \underline{c}D  \\
C &\rightarrow \underline{c}D  \\
D &\rightarrow \epsilon  \\
E &\rightarrow \underline{a}A
\end{align*}


\subsection*{b.)}
\begin{align*}
L &= \{\underline{a}^{(2n)} \underline{b} \; \underline{c}^* \; (\underline{bb} | \underline{d}) \; | n>0 \}\\
S &\rightarrow \underline{a}A  \\
A &\rightarrow \underline{a}A  \\
A &\rightarrow \underline{b}C  \\
A &\rightarrow \underline{b}B  \\
A &\rightarrow \underline{b}D  \\
B &\rightarrow \underline{b}E  \\
B &\rightarrow \epsilon  \\
C &\rightarrow \underline{c}C  \\
C &\rightarrow \underline{c}B  \\
C &\rightarrow \underline{c}D  \\
D &\rightarrow \underline{d}D  \\
D &\rightarrow \epsilon  \\
E &\rightarrow \underline{b}B
\end{align*}

\section*{Beispiel 2}
\subsection*{a.)}
Dies ist eine \textbf{allgemeine} Grammatik da $ \mid\alpha\mid \leq \mid\beta\mid$ \textbf{nicht} gilt und somit keine Restriktion $\alpha \rightarrow \beta$  gilt 
\subsection*{b.)}
Dies ist eine \textbf{reguläre} Grammatik da $ \mid\alpha\mid \leq \mid\beta\mid , \alpha \in$ V\textsubscript{N} 
$\beta$ hat form aA oder a, mit \\a $\in$ V\textsubscript{T} $\bigcup$ \{$\epsilon$\}, A $\in$ V\textsubscript{N}
\subsection*{c.)}
Dies ist eine \textbf{kontextfreie} Grammatik da  $\mid\alpha\mid \leq \mid\beta\mid\, , \alpha \in$ V\textsubscript{N}
\subsection*{d.)}
Dies ist keine \textbf{gültige} Grammatik da $R \rightarrow Q\underline{y} $ nicht laut Definition $\alpha \, \beta \in (V\textsubscript{N} \cup V\textsubscript{T})$ diese Form nicht in der Grammatik definiert ist.
\subsection*{e.)}
Dies ist eine \textbf{allgemeine} Grammatik da $ \mid\alpha\mid \leq \mid\beta\mid$ \textbf{nicht} gilt und somit keine Restriktion $\alpha \rightarrow \beta$  gilt 
\subsection*{f.)}
Dies ist keine \textbf{gültige} Grammatik da $\underline{num} \rightarrow      \underline{var}$ nicht in den Grammatik definiert ist



\section*{Beispiel 3}
\subsection*{First und Follow Mengen:}

\begin{table}[htbp]
    \centering
      \begin{tabular}{|m{1cm}|m{2cm}|m{2cm}|}
\hline
 &FIRST & FOLLOW \\ \hline
S & \underline{a} \underline{b} \underline{c} \underline{d} \underline{e} & \$ \\ \hline
A & \underline{a}  \underline{b} \underline{c} &  \underline{d} \underline{e} \\ \hline
B & \underline{d} \underline{e} & \underline{c} \underline{d} \underline{e}\\ \hline
C & \underline{b}  & \underline{c} \underline{d} \underline{e} \\ \hline
D & \underline{c} & \underline{d} \underline{e} \\ \hline
E & \underline{d} \underline{e} & \underline{c} \underline{d} \underline{e}\\ 
\hline
        \end{tabular}
\end{table}

\newpage
\subsection*{LL(1) Tabelle}

\begin{table}[htbp]
    \centering
      \begin{tabular}{|c|c|c|c|c|c|c|}
\hline
   & \underline{a} & \underline{b} & \underline{c} & \underline{d} & \underline{e} & \$ \\ \hline
S & S $\rightarrow$ AB & S $\rightarrow$ AB & S $\rightarrow$ AB&  S $\rightarrow$ AB  &  S $\rightarrow$ AB &  S $\rightarrow$ AB \\ \hline
A & A $\rightarrow$ \underline{a}A & A $\rightarrow$ CD & A $\rightarrow$ CD &   &  &  \\ \hline
B &  &   &   & B $\rightarrow$ E &B $\rightarrow$ E  &   \\ \hline
C  &  & C $\rightarrow$ \underline{b} B & C $\rightarrow \epsilon$ &C $\rightarrow \epsilon$   & C $\rightarrow \epsilon$ & \\ \hline
D &  &   &D $\rightarrow$ \underline{c}C   &D $\rightarrow \epsilon$   &D $\rightarrow \epsilon$  &   \\ \hline
E &  &   & &E $\rightarrow$ \underline{d}   &E $\rightarrow$ \underline{e}  &   \\
\hline
        \end{tabular}
\end{table}

\section*{Beispiel 4}

Gegeben ist die folgende LL(1) Tabelle, welche eine grobe Abstraktion der Variablendeklaration in Scala beschreibt. Die unterstrichenen Zeichenketten in den Spalten der ersten Zeile stellen jeweils ein Terminalsymbol dar.\\
\\
\begin{center}
\begin{tabular}{ |c|c|c|c|c|c|c|c|c|c|c|c| }
 \hline
 &\underline{var} &\underline{val} &\underline{one} &\underline{two} &\underline{String} &\underline{Int} &\underline{0} &\underline{1} &\underline{$"$} &\underline{=} &\$\\ 
 \hline
 S &AB &AB & & & & & & & & & \\ 
  \hline
 A &CN\underline{$:$} &CN\underline{$:$} & & & & & & & & & \\
  \hline
 B & & & & &\underline{String}\underline{=}\underline{$"$}\underline{$V$}\underline{$"$} &\underline{Int}\underline{=}\underline{U} & & & & & \\
  \hline
 C &\underline{var} &\underline{val} & & & & & & & & & \\
  \hline
 N & & &\underline{one} &\underline{two} & & & & & & &  \\
  \hline
 U & & & & & & &\underline{0}V &\underline{1}V & & &\\
  \hline
V & & & & & & &\underline{0}V &\underline{1}V &$\varepsilon$ & &$\varepsilon$\\
   \hline
 \end{tabular}\\
 \end{center}
 %\\
% \\
 \"Uberpr\"ufen Sie mittels der gegebenen LL(1) Tabelle ob folgende Ausdr\"ucke g\"ultige S\"atze der definierten
Grammatik sind:%\\
%\\
a)\underline{var} \underline{one} \underline{:} \underline{String} \underline{=} \underline{1}\\
b)\underline{val} \underline{two} \underline{:} \underline{int} \underline{=} \underline{10}\\
%\\
Die L\"osung f\"ur die Unterpunkte a und b soll im folgenden Format erarbeitet und abgegeben werden:\\
\begin{center}
\begin{tabular}{ |c|c|c|  }
 \hline
 Stack&Input &Produktion/Kommentar\\ 
 \hline
 \$S &val one: Int= "11"\$ &... \\
  \hline
 \end{tabular}\\
 \end{center}
% \\

\subsection*{a.)}
%\\
%\\
\begin{tabular}{  |c|c|c|  }
 \hline
 Stack&Input &Produktion/Kommentar\\ 
 \hline
 \$S &var one : String = 1 \$ &S:= AB \\
  \hline
 \$BA &var one : String = 1\$ &A:= CN\underline{:}\\
 \hline
 \$B\underline{:}NC &var one : String = 1 \$ &C:= \underline{var}\\
 \hline
 \$B\underline{:}N\st{var} &\st{var} one : String = 1\$ &\\
   \hline
 \$B\underline{:}N &one : String= 1\$ &N:= \underline{one} \\
   \hline
 \$B\st{:}\st{one} &\st{one}\st{:} String= 1\$ & \\
   \hline
 \$B & String = 1\$ & B:= \underline{String}\underline{=}\underline{"}V\underline{"} \\
   \hline
 \$"V"\st{=}\st{String} &\st{String}\st{=}1\$ & \\
   \hline
 \$"V" &1\$ & Keine Regel \\
   \hline
 
 \end{tabular}\\
 \\
 Diese Satz ist nicht Valid!
 \\
 \subsection*{b.)}
 %\\
\begin{tabular}{ |c|c|c|  }

 \hline
 Stack&Input &Produktion/Kommentar\\ 
 \hline
 \$S &val two : int = 10\$ &S:= AB \\
  \hline
 \$BA &val two : int= 10\$ &A:= CN\underline{:}\\
 \hline
 \$B\underline{:}NC &val two : int= 10\$ &C:= \st{val}\\
 \hline
 \$B\underline{:}N\st{val} &\st{val} two : int = 10\$ &\\
   \hline
 \$B\underline{:}N &two : int = 10\$ &N:=\underline{two} \\
   \hline
 \$B\st{:}\st{two} &\st{two}\st{:} int = 10\$ & \\
   \hline
 \$B &int= 10\$ & B:= \underline{int}\underline{=}U \\
   \hline
 \$U\st{=}\st{int} &\st{int}\st{=} 10\$ & \\
   \hline
 \$U  &10\$ & U:=\underline{1}V \\
   \hline
 \$V\st{1} &\st{1}0\$ &\\
   \hline
 \$V &0\$ &V:= \underline{0}V\\
   \hline
 \$V\st{0}  &\st{0}\$ & \\
   \hline
 \$V  &\$  &V:= $\epsilon$\\
   \hline
 \$ &\$ &Valid \\
 \hline

 \end{tabular}\\
Diese Satz ist Valid!
\\

\newpage
\section*{Beispiel 5}
Die gegebene Gramatik ist keine LL(1) Gramatik weil:
\begin{itemize}
\item X ist Mehrdeutig (Linksfaktorisierung)
\item Y ist indirekt Links recrusiv (über Z)
\end{itemize}


\subsection*{\"Aquivalente LL(1) Grammatik:}
\begin{align*}
S &\rightarrow X Y \\
V &\rightarrow \underline{x} \\
V &\rightarrow \underline{-} \\
W &\rightarrow \underline{0} W \\
W &\rightarrow \underline{1} W \\
W &\rightarrow \underline{\epsilon} \\
X &\rightarrow \underline{x} X_2 \\
X_2 &\rightarrow V \\
X_2 &\rightarrow W \\
Y &\rightarrow \underline{+} Y_2 \\
Y_2 &\rightarrow \underline{y} W Y_2 \\
Y_2 &\rightarrow \underline{\epsilon} \\
\end{align*}

\subsection*{LL(1) Tabelle}
\begin{tabular}{|c|c|c|c|c|c|c|c|}
\hline 
 & \underline{x} & \underline{-} & \underline{0} & \underline{1} & \underline{+} & \underline{y} & \$ \\ 
\hline 
$S$ & $S \rightarrow X Y $&   &   &   &   &   &   \\ 
\hline 
$V$ & $V \rightarrow X$ & $V \rightarrow \underline{-} $&   &   &   &   &   \\ 
\hline 
$W$ &   &   & $W \rightarrow \underline{0} W $& W $\rightarrow \underline{1} W $&$ W \rightarrow \epsilon$ & W $\rightarrow \epsilon$ & $W \rightarrow\epsilon$ \\ 
\hline 
$X$ & $X \rightarrow \underline{x} X_2 $&   &   &   &   &   &   \\ 
\hline 
$X_2$ & $X_2 \rightarrow V $& $X_2 \rightarrow V$ & $X2 \rightarrow W $& $X2 \rightarrow W $& $X2 \rightarrow W $&  &  \\ 
\hline 
$Y$ &   &   &   &   & $Y \rightarrow \underline{+} Y_2 $&   &   \\ 
\hline 
$Y_2$ &   &   &   &   &   & $Y2 \rightarrow \underline{y} W Y_2 $& $Y_2 \rightarrow \epsilon$ \\
\hline 
\end{tabular} 

\newpage
\section*{Beispiel 6}
\subsection*{a.)}
\begin{lstlisting}[language=scala]
object SimpleParserA extends JavaTokenParsers {
  def A: Parser[Any] = "aa" ~ A | "c" ~ ""
  def B: Parser[Any] = "b" ~ B | "c" ~ ""
  def AB: Parser[Any] = A | B

  def S: Parser[Any] = "a" ~ AB | "ac"
  def apply(s: String) = parseAll(S, s)
}
\end{lstlisting}
\subsection*{a.)}
\begin{lstlisting}[language=scala]
object SimpleParserB extends JavaTokenParsers {
  def N: Parser[Double] = "[1-9][0-9]*|0|-[1-9][0-9]*".r 
                          ^^ {i => i.toDouble}
  def S: Parser[Double] = N~"*10^"~N ^^ {case a~_~b => a*math.pow(10,b)}
  def apply(s: String) = parseAll(S, s)
}
\end{lstlisting}
\end{document}

